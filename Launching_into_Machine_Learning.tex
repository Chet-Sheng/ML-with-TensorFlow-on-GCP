\chapter{Launching into Machine Learning}
\section{Supervised Learning}
\subsection{Linear Regression}
Linear regression有closed form solution. 或者也可以用gradient descent.

\subsubsection{Closed Form Solution}
However, closed form solution is not practical for large dataset.
\[ W=(X^T X)^{-1} X^T Y \]
\begin{itemize}
	\item $ X^T X $首先就要是non-singular matrix. ie.行列都要是linear independent的.不然不能inverse.
	\item 计算复杂度也很高,$ O(n^3) $.
\end{itemize}

\subsubsection{Gradient Descent}
\begin{itemize}
	\item less expensive computationally in both time and memory;
	\item more amenable to model generalization;
	\item generic enough to work on most problems.
\end{itemize}

\subsection{Neural Networks}
\subsubsection{Actication Functions}
\begin{itemize}
	\item Sigmoid/Tanh: gradient descent halt/ slow down when its value close to 1 or -1.
	\item ReLu: it is non-linear, so we can get the complex modeling needed.\\
	It doesn't have the saturation in non-negative portion of input space.\\
	However, the negative portion of input space translating to a zero activation, ReLu layers could end up dying or no long er activating, which can also cause traning to slow or stop.
	\item ELU: Solve the ReLU problem, but computational expensive to calculate exponential.
	
\end{itemize}
When we need a probability as the output of a model, the activation function of the last layer should be sigmoid: 
\begin{itemize}
	\item  Beyond just the range, the sigmoid(logistic) function is the cumulative distribution function(CDF) of the logistic probability distribution.\\
	Quantile function of logistic probability distribution (the inverse of CDF) is a generalization of the logit function\footnote{logit: \url{https://en.wikipedia.org/wiki/Logit}}, which models the log odds when the function's variable represents a probability $ p $.\\
	This is why it can be used as a true probability.\\
	(卧槽,这玩意儿解决了我的一个万年疑问啊!怎么感觉还是不是特别清楚,还是要看看怎么定义的概率)
	\item  Tanh is incorrect, because even though it is a squashing function like a sigmoid, its range ranges between negative one to one which is not the same range as the probability. \\
	Furthermore, just squashing tanh into a sigmoid will not magically turn it into a probability because it doesn't have the same properties mentioned above that allows a sigmoid output to be interpreted as a probability.\\
	To correctly convert into a sigmoid:
	\begin{itemize}
		\item first you have to add one and divide by two to get the correct range.
		\item Also, to get the right spread, you'd have to divide tanh argument by two. 
	\end{itemize} 
	But you've already calculated tanh, so we will pinned a bunch of work, and you may as well have just used a sigmoid to start.
\end{itemize}

 