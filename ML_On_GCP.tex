% use following default compiler:
% txs:///xelatex  
\documentclass[9pt,twocolumn,oneside,a4paper]{book}
\usepackage[a4paper, 
left=15mm,
right=15mm,
top=20mm,
bottom=20mm,
footskip=10mm
]{geometry}
%========================================================
% Basic Setting
\setlength{\columnsep}{0.5cm} % two columns seperation distance
\usepackage{color}
\usepackage{amsmath}
\usepackage{mathtools}
\DeclareMathOperator*{\argmin}{argmin} % no space, limits underneath in displays
\DeclareMathOperator*{\argmax}{argmax} % no space, limits underneath in displays

\usepackage{amsfonts}
\usepackage{bm}
\usepackage{relsize}
\usepackage{amssymb}
\usepackage{makeidx}
\usepackage{graphicx}
\usepackage{lmodern}
\setlength{\parindent}{0pt}  % set indent to be 0
%\usepackage[bottom]{footmisc}  % places footnotes at page bottom

\usepackage{float} % graph position related package
\definecolor{codegreen}{rgb}{0,0.6,0}
\definecolor{codegray}{rgb}{0.5,0.5,0.5}
\definecolor{codepurple}{rgb}{0.58,0,0.82}
\definecolor{backcolour}{rgb}{0.95,0.95,0.92}

%========================================================
% Code format
\usepackage{listings}
\lstdefinestyle{mystyle}{
	backgroundcolor=\color{backcolour},   
	commentstyle=\color{codegreen},
	keywordstyle=\color{magenta},
	numberstyle=\tiny\color{codegray},
	stringstyle=\color{codepurple},
	basicstyle=\footnotesize,
	breakatwhitespace=false,         
	breaklines=true,                 
	captionpos=b,                    
	keepspaces=true,                 
	numbers=left,                    
	numbersep=5pt,                  
	showspaces=false,                
	showstringspaces=false,
	showtabs=false,                  
	tabsize=2
}
%========================================================
% Customized Setting

% 1. Additional color library for more color:
\usepackage[dvipsnames]{xcolor} 

% 2. Underline setting:
\usepackage{soul}  % this is for underlining with line change. Type: \ul{}
\setul{3pt}{0.5pt}

% Explanation:
%\setul{underline depth}{underline thickness} 
% depth: how close the line is to the words
% thickness: how thick the line is
%========================================================
% Check mark and x mark Setting

% check mark is Tick or Nike
\usepackage{amssymb}% http://ctan.org/pkg/amssymb
\usepackage{pifont}% http://ctan.org/pkg/pifont
\newcommand{\cmark}{\ding{51}}%
\newcommand{\xmark}{\ding{55}}%

%========================================================
% Images or Picture Setting: enable insert images occupy 2 columns
%\newenvironment{figurehere}
%{\def\@captype{figure}}
%{}

%========================================================
% Type Chinese
\usepackage{xeCJK}
\usepackage{fontspec} 
%\setCJKmainfont{WenQuanYi\ Mincro\ Hei}
%\setCJKmainfont{STKaiti}
\setCJKmainfont{Source Han Sans SC Normal}
%\setCJKmainfont{Songti SC}
%\setCJKmainfont{AR\ PL\ UKai\ CN}
%========================================================
% Fancy Header and Footer
\usepackage{fancyhdr}
\usepackage{lipsum}% just to generate text for the example

\pagestyle{fancy}
\fancyhf{}     % sets all head and foot elements empty
\fancyhead[L]{\leftmark}  
%sets the left head element to \rightmark, which contains the current section (\leftmark is the current chapter).
\fancyhead[R]{\rightmark}  
%\fancyhead[R]{\thepage}  % sets the right head element to the page number

\fancyfoot[C]{\thepage}

\renewcommand{\headrulewidth}{0.5pt}  % lets the head rule disappear.
%========================================================
% Chapter Title Formating (not for Part Page format)
\usepackage{titlesec}
\titleformat{\chapter}[display]
{\normalfont\Huge \scshape}{\titlerule}{-28pt}{\thechapter\ \ \Huge}[\vspace{-2pt}\titlerule]
\titleformat{name=\chapter,numberless}[display]
{\normalfont\Huge \scshape}{\titlerule}{-28pt}{\Huge}[\vspace{-2pt}\titlerule]
\titlespacing*{\chapter}{0pt}{0pt}{20pt}
%========================================================

\usepackage[justification=centering]{caption}
\usepackage{subfig}
\usepackage{multirow}
\usepackage[
bookmarksnumbered=true,
bookmarksopen=true,
bookmarks=true,
bookmarksopenlevel=1,
colorlinks=true,
linkcolor=blue,
anchorcolor=blue,
citecolor=blue
]{hyperref}

\graphicspath{{figures/}}


%\renewcommand{\chaptermark}[1]{\markright{\thechapter .\ #1}}
%\renewcommand{\subsectionmark}[1]{\markright{\thesubsection\ #1}}
%========================================================
% Use definition and theotem blocks
\usepackage{amsthm}
\theoremstyle{plain}
%\newtheorem{thm}{Theorem}[section]
\newtheorem{thm}{Thm}[section]
\newtheorem{lem}[thm]{Lemma}
\newtheorem{prop}[thm]{Proposition}
\newtheorem*{cor}{Corollary}

\theoremstyle{definition}
%\newtheorem{defn}{Definition}[section]
\newtheorem{defn}{Def}[section]
\newtheorem{conj}{Conjecture}[section]
%\newtheorem{exmp}{Example}[section]
\newtheorem{exmp}{Exmp}[section]

\theoremstyle{remark}
\newtheorem*{rem}{Remark}
\newtheorem*{note}{Note}

%========================================================
% insert pdf in doc
\usepackage{pdfpages}

%========================================================
% Code Blocks library
\usepackage{listings}

\definecolor{mygreen}{rgb}{0,0.6,0}
\definecolor{mygray}{rgb}{0.5,0.5,0.5}
\definecolor{mymauve}{rgb}{0.58,0,0.82}

\lstset{ 
	backgroundcolor=\color{white},   % choose the background color; you must add \usepackage{color} or \usepackage{xcolor}; should come as last argument
	basicstyle=\footnotesize,        % the size of the fonts that are used for the code
	breakatwhitespace=false,         % sets if automatic breaks should only happen at whitespace
	breaklines=true,                 % sets automatic line breaking
	captionpos=b,                    % sets the caption-position to bottom
	commentstyle=\color{mygreen},    % comment style
	deletekeywords={...},            % if you want to delete keywords from the given language
	escapeinside={\%*}{*)},          % if you want to add LaTeX within your code
	extendedchars=true,              % lets you use non-ASCII characters; for 8-bits encodings only, does not work with UTF-8
	frame=single,	                   % adds a frame around the code
	keepspaces=true,                 % keeps spaces in text, useful for keeping indentation of code (possibly needs columns=flexible)
	keywordstyle=\color{blue},       % keyword style
	language=Octave,                 % the language of the code
	morekeywords={*,...},            % if you want to add more keywords to the set
	numbers=left,                    % where to put the line-numbers; possible values are (none, left, right)
	numbersep=5pt,                   % how far the line-numbers are from the code
	numberstyle=\tiny\color{mygray}, % the style that is used for the line-numbers
	rulecolor=\color{black},         % if not set, the frame-color may be changed on line-breaks within not-black text (e.g. comments (green here))
	showspaces=false,                % show spaces everywhere adding particular underscores; it overrides 'showstringspaces'
	showstringspaces=false,          % underline spaces within strings only
	showtabs=false,                  % show tabs within strings adding particular underscores
	stepnumber=2,                    % the step between two line-numbers. If it's 1, each line will be numbered
	stringstyle=\color{mymauve},     % string literal style
	tabsize=2,	                   % sets default tabsize to 2 spaces
	title=\lstname                   % show the filename of files included with \lstinputlisting; also try caption instead of title
}
%========================================================

\begin{document}
	\frontmatter%%%%%%%%%%%%%%%%%%%%%%%%%%%%%%%%%%%%%%%%%%%%%%%%%%%%%%
	\title{\textbf{Machine Learning with Tensorflow\\on Google Cloud Platform}}
	\author{Huafeng Sheng}
	%\date{December 2004}
	\maketitle
	\tableofcontents
	
	% Introductory chapters
	\chapter{Preface}
	This is a personal notebook for Coursera Course: 
	
	Machine Learning with Tensorflow on Google Cloud Platform
	
	\mainmatter%%%%%%%%%%%%%%%%%%%%%%%%%%%%%%%%%%%%%%%%%%%%%%%%%%%%%%
%	\part{Math}
	\section{Module 1}
sadf asd f
\includepdf[pages=-]{figures/Module1.pdf}[H]
asdf sad f

%\begin{figure}[h]
%	\centering
%	\includegraphics{figures/Module1.pdf}
%	\caption{Schematical view of Spearman's theory.}
%\end{figure}


\begin{figure}[H]
	\centering
	\includegraphics[width=0.5\textwidth]{figures/Module1.pdf}
	\caption{LSTM unfolded through time steps}
\end{figure}
	\chapter{Launching into Machine Learning}
\section{Supervised Learning}

\begin{figure}[H]
	\centering
	\includegraphics[width=0.5\textwidth]{figures/MLShortHistory}
	\caption{ML Short History}
\end{figure}

\subsection{Linear Regression}

他有个数据可视化的分析方法也挺好的,就是数据量太大的时候scatter plot就不合适了,他用了画quantile的方法,在两坐标轴不变的情况下,让图像更加可视化。卧槽,好思路啊!我惊呆了。
\begin{figure}[H]
	\centering
	\includegraphics[width=0.4\textwidth]{figures/BabyWeight-vs-MotherAge(Scatter)}
	\caption{Scatter Plot}
\end{figure}
\begin{figure}[H]
	\centering
	\includegraphics[width=0.4\textwidth]{figures/BabyWeight-vs-MotherAge(Quantile)}
	\caption{Quantile Plot}
\end{figure}



Linear regression有closed form solution. 或者也可以用gradient descent.

\subsubsection{Closed Form Solution}
However, closed form solution is not practical for large dataset.
\[ W=(X^T X)^{-1} X^T Y \]
\begin{itemize}
	\item $ X^T X $首先就要是non-singular matrix. ie.行列都要是linear independent的.不然不能inverse.
	\item 计算复杂度也很高,$ O(n^3) $.
\end{itemize}
数据量太大的时候Analytical solution(上面提到的)become impractical. 我们就会改用 Gradient Descent的方法。


\subsubsection{Gradient Descent}
\begin{itemize}
	\item less expensive computationally in both time and memory;
	\item more amenable to model generalization;
	\item generic enough to work on most problems.
\end{itemize}

\subsection{Neural Networks}
\subsubsection{Actication Functions}
\begin{itemize}
	\item Sigmoid/Tanh: gradient descent halt/ slow down when its value close to 1 or -1.
	\item ReLu: it is non-linear, so we can get the complex modeling needed.\\
	It doesn't have the saturation in non-negative portion of input space.\\
	However, the negative portion of input space translating to a zero activation, ReLu layers could end up dying or no long er activating, which can also cause traning to slow or stop.
	\item ELU: Solve the ReLU problem, but computational expensive to calculate exponential.
	
\end{itemize}
When we need a probability as the output of a model, the activation function of the last layer should be sigmoid: 
\begin{itemize}
	\item  Beyond just the range, the sigmoid(logistic) function is the cumulative distribution function(CDF) of the logistic probability distribution.\\
	Quantile function of logistic probability distribution (the inverse of CDF) is a generalization of the logit function\footnote{logit: \url{https://en.wikipedia.org/wiki/Logit}}, which models the log odds when the function's variable represents a probability $ p $.\\
	This is why it can be used as a true probability.\\
	(卧槽,这玩意儿解决了我的一个万年疑问啊!怎么感觉还是不是特别清楚,还是要看看怎么定义的概率)
	\item  Tanh is incorrect, because even though it is a squashing function like a sigmoid, its range ranges between negative one to one which is not the same range as the probability. \\
	Furthermore, just squashing tanh into a sigmoid will not magically turn it into a probability because it doesn't have the same properties mentioned above that allows a sigmoid output to be interpreted as a probability.\\
	To correctly convert into a sigmoid:
	\begin{itemize}
		\item first you have to add one and divide by two to get the correct range.
		\item Also, to get the right spread, you'd have to divide tanh argument by two. 
	\end{itemize} 
	But you've already calculated tanh, so we will pinned a bunch of work, and you may as well have just used a sigmoid to start.
\end{itemize}


\subsection{Decision Tree}
each node is  a linear classifier for one feature.

\subsection{Support Vector Machine(SVM)} 
maximise the margin between two support vectors, and minimize the hinge loss.
 
 In case of having more than 2 classes, a one vs all approach should be taken and then choose the best out of the promoted binary classification.
 
 In case of features non-linear separable, we apply kernel transformation which maps the data from our input vector space to a vector space that now has features that can be linearly separated as shown in the diagram. 
 kernel transformation 类似神经网络的激活函数把input变成transform space, output dimension is decided by number of neurons in the layer. The activation function changes the basis of the vector space but doesn't addor subtract dimensions.  
 
 Kernelized SVMs tend to provide sparser solutions and thus have better scalability. 
 SVMs perform better when there is a high number of dimensions and when the predictors nearly certainly predict the response.
 
 \subsection{Random forest}
 	blend performance across many models, and we call this ensemble method.
 	eg. if errors are independent for a number of simple weak learners, combined, they would form a strong learner.
 	DNN approximate this by using dropout layers, which help regularize the model and prevent overfitting. This can be simulated by randomly turning off neurons in the network with some probabilities for each forward pass, which will essentially be creating a new network each time.
 
\subsection{DNN}
 	Back again in the timeline are neural networks, now with even more of an advantage through leaps and computational power and lots and lots of data. DNNs began to substantially outperform other methods on test such as computer vision. In addition to the boom from boosted hardware, there are many new tricks and architectures that help to improve trainability of deep neural networks like ReLUs, better initialization methods, CNNs or convolutional neural networks and dropout. 
 	We talked about some of these tricks from other ML methods. The use of nonlinear activation functions such as ReLUs, which usually are set as the default now, we talked about during the first look at neural networks. 
 	Dropout layers began being used to help with generalization, which works like ensemble methods, which we explored when talking about random forests and boosted trees. 
 	Convolutional layers were added that reduced the computational and memory load due to their non-complete connectedness, as well as being able to focus on local aspects, for instance, images, rather than comparing unrelated things in an image. 
 	
 	In other words, all the advances that came about in other ML methods, got folded back into neural networks. Let's look at an example of the deep neural network
 	
\section{Optimisation}
\subsection{Loss function}
loss functon,作用是measure the quality of a group of predictions on training data (sum of a group of prediction error).

linear regression不仅可以用作regression,还可以做classification,有个东西叫做decision boundry,就像svm一样画条线分开。。。

然后loss function for linear regression我们就用RMSE (Root Mean Square Error\footnote{RMSE: 之所以要加Root是因为计量单位问题}).

但是for classification, 我们就不再用RMSE了,除了deeplearning里面提到的问题,还有几个问题:
\begin{itemize}
	\item The problem is, it fails to capture our intuitive belief that	predictions that are really bad should be penalized much more strongly. Note how a prediction of one, when the target is zero, is about three times worse than a prediction of point five for the same target.
	\item Cross-Entropy Loss penalize bad prediction very strongly.
\end{itemize}
\begin{figure}[H]
	\centering
	\includegraphics[width=0.4\textwidth]{figures/RMSE-for-Classification}
	\caption{RMSE for Classification}
\end{figure}
\begin{figure}[H]
	\centering
	\includegraphics[width=0.4\textwidth]{figures/CrossEntropyLoss-for-Classification}
	\caption{Cross-Entropy Loss for Classification}
\end{figure}



	\medskip
%	
%	\bibliographystyle{unsrt}
%	\bibliography{bibliography} % put name of .bib file in {}.
%	\appendix
%	\chapter{First Appendix}
	
	\backmatter%%%%%%%%%%%%%%%%%%%%%%%%%%%%%%%%%%%%%%%%%%%%%%%%%%%%%%
%	\chapter{Last note}
	%	\begin{abstract}
	%		Your abstract goes here...
	%	\end{abstract}
	
\end{document}